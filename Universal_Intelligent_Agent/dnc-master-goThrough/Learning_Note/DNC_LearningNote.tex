\documentclass{article}
\title{Differentiable Neural Computer Learning Note}
\author{Zhengzhong Liang}
\date{2018-04-30}

\begin{document}
\maketitle
\section{Code Structure}
\subsection{About Computation Graph}
The basic graph is defined in ``DNC.py'', some peripheral parts of graph is defined in ``run$\_$model''function and ``train.py''. The main parts of computation graph of DNC is organized as following:
\begin{table}
\centering
\caption{Organization of DNC}
\begin{tabular}{| l | c | c | c | r |} \hline
name & functionality & script & function & class \\ \hline
controller & an LSTM network which reads data & DNC.py & x & DNC \\ \hline
\end{tabular}
\end{table}

\subsection{Activities in Each Loop}
In each loop of training in iteration, the agent does the following things: 
\paragraph{Some input is presented}
\paragraph{Input is given to LSTM}
\paragraph{Something is written to memory}
\paragraph{Something is output from memory}
\paragraph{The output is decoded by LSTM and printed}


\section{Topics to Study}
\subsection{What type of task does this agent try to finish}
\subsection{What is the function of memory in this task}
\subsection{What together tasks can we train the agent to do}
\subsection{What parts and functions can we add to this agent}

\section{Question}
\subsection{The memory and controller are defined, but are not connected}
\subsection{The build function is not used}
\end{document}